% Settings for the default beamer theme
\documentclass[english, aspectratio=169]{beamer}
\usepackage[T1]{fontenc}
\usepackage[utf8]{inputenc}
\usepackage{tabularx}
\usepackage{babel}
\usepackage[ruled,vlined]{algorithm2e}
\SetAlgorithmName{Algoritmus}{algoritmus}{List of Algorithms}
\setcounter{secnumdepth}{3}
\setcounter{tocdepth}{3}

\makeatletter

\newcommand\makebeamertitle{\frame{\maketitle}}

% (ERT) argument for the TOC
\AtBeginDocument{%
  \let\origtableofcontents=\tableofcontents
  \def\tableofcontents{\@ifnextchar[{\origtableofcontents}{\gobbletableofcontents}}
  \def\gobbletableofcontents#1{\origtableofcontents}
}

% Theme settings
\usetheme{Frankfurt}
\usecolortheme{default}
\usefonttheme[onlymath]{serif}

% Template settings
\setbeamertemplate{navigation symbols}{}
\setbeamertemplate{blocks}[rounded][shadow=false]
\setbeamertemplate{title page}[default][colsep=-4bp, rounded=true, shadow=false]
\makeatother

% Define a custom darker red color
\definecolor{DarkerRed}{RGB}{139,0,0} % Adjust the RGB values as needed

% Use the newly defined color in Beamer theme elements
\setbeamercolor{structure}{fg=DarkerRed} % Changes basic structural elements to Darker Red
\setbeamercolor{title in head/foot}{bg=DarkerRed} % Changes the title in header/footer to Darker Red


\begin{document}

% Title page
\section{Bevezetés}
\title[]{Üzleti Elemzések Módszertana}
\subtitle{Tantárgyi útmutató}
\author[Kuknyó Dániel]{Kuknyó Dániel\\Budapesti Gazdasági Egyetem}
\date{2023/24\\2.félév}
\makebeamertitle

\begin{frame}{Elérhetőségek}
\begin{itemize}
	\item E-mail: \href{mailto:daniel.kuknyo@mailbox.org}{daniel.kuknyo@mailbox.org}
	\item Messenger: \href{https://www.facebook.com/dani.kkny/}{Dani Kuknyo}
	\item A tantárgy Git tárhelye: \href{https://github.com/basictask/Elemzesmodszertan}{basictask/elemzesmodszertan}
	\item Kérek mindenkit, Teams és Coospace felületen ne írjon, mert nem olvasom rendszeresen. 
	\item Bármilyen kérdéssel és problémával keressetek nyugodtan. 
\end{itemize}
\end{frame}

\begin{frame}{A tantárgy tematikája}
\begin{enumerate}
	\item Regressziós eljárások
	\item Osztályozási eljárások
	\item Regularizált lineáris modellek
	\item Döntési fák
	\item Együttes tanuló algoritmusok
	\item Tartó vektor gépek
	\item Felügyelet nélküli tanulás
	\item Generatív modellezés
	\item Ajánló rendszerek
	\item Neurális hálózatok 
	\item Megerősítéses tanulás
\end{enumerate}
\end{frame}

\begin{frame}{A jegy összetétele}
\begin{itemize}
	\item Összesen 100p gyűjthető
	\item Két elméleti teszt, egyenként 15-15p: az előadáson elhangzottakból
	\item 70p csoportos projekt feladat:
	\begin{itemize}
		\item Max. 3 fős csapatok
		\item Téma kiválasztása és kidolgozása
		\item A projekt összetétele:
		\begin{itemize}
			\item Jelenléti előadás
			\item Elemzési csővezeték (program)
			\item Dokumentáció
		\end{itemize}
	\end{itemize}
	\item Az órákon a jelenlét kötelező. 
\end{itemize}
\end{frame}

\begin{frame}{Ponthatárok}
\begin{itemize}
	\item A teljes pontszám az elméleti tesztek és a beadandók összege
	\item Ponthatárok:\\
	$90 \leq x \leq 100 \rightarrow 5$\\
	$80 \leq x < 90 \rightarrow 4$\\
	$70 \leq x < 80 \rightarrow 3$\\
	$60 \leq x < 70 \rightarrow 2$\\
	$x < 60 \rightarrow 1$
	\item Ha a Hallgatónak nem sikerül elérnie a minimum követelményt, tehet vizsgát.
\end{itemize}
\end{frame}

\begin{frame}{A téma kidolgozása}
\begin{itemize}
	\item A felsorolt lehetőségek közül minden csapatnak választania kell egy témát, majd egy megfelelő adathalmazt felhasználva az órán tanult módszerek segítségével elemzést készíteni. 
	\item A témáktól el lehet térni olyan algoritmusokra, modellfajtákra, amelyek az óra tárgyát nem képezték, de a módszertani témán belül vannak. 
	\item \textbf{A munka egyedisége és eredetisége elvárt és ellenőrzött.}
	\item Konzultációs időre lehetősége van minden csapatnak, ahol feltehetik a kérdéseiket, tanácsot kérhetnek. Ez kb. 20-30 perc, Teams felületen.
	\item Ha nem teljesen világos a projekt terjedelme (mit, mennyit, milyen részletesen...) konzultáció keretein belül közösen segítünk meghatározni.
\end{itemize}
\end{frame}

\begin{frame}{A téma kidolgozása}
\begin{itemize}
	\item Terjedelem: min. 4-5 oldal képek nélkül.\par\smallskip
	\item Tartalma:
	\begin{itemize}
		\item Az elemzés célja, megválaszolandó kérdések
		\item Adatforrások bemutatása, tisztítási módszerek és ezek elméleti alapjai
		\item A felhasznált módszerek és ezek elméleti alapjai
		\item A hiperparaméter optimalizálás módszerei és eredményei
		\item A kutatási eredmények ismertetése, értelmezése és vizuális bemutatása
		\item Javaslatok és további fejlesztés lehetőségei, más rendszerekkel való kapcsolatok
	\end{itemize}\par\smallskip
	\item A beadandó feladat feltöltésekor csatolni kell (csapatonként 1 embernek):
	\begin{itemize}
		\item A felhasznált adatokat, és a forrást, ahonnan az adatok származnak [\texttt{.csv, .xlsx}]
		\item A programkódot, ami az elemzést megvalósítja [\texttt{.py, .ipynb stb...}]
		\item Az elemzéshez tartozó dokumentációt [\texttt{.pdf}]
		\item Az elemzésről szóló bemutatót [\texttt{.pptx, .pdf stb...}]
	\end{itemize}
	\item \textbf{Minden külsőleg felhasznált anyagot, cikket le kell hivatkozni. A nem hivatkozott külső forrásból származó tartalom plágiumnak minősül.} 
\end{itemize}
\end{frame}

\begin{frame}{Választható témák}
\begin{enumerate}
	\item \textbf{Regresszió} [Lineáris, Logisztikus] és Gradiens ereszkedés [Kötegelt, Sztochasztikus, Mini-kötegelt]
	\item \textbf{Döntési fák, Erdő modellek} [Bagging, Pasting, optimalizálás] Turbózás, együttes tanulás [Adaboost, GBM, XGBoost]
	\item \textbf{Regularizált modellek} [Lasso, Ridge, Elasztikus hálók]
	\item \textbf{Tartó vektor gépek} [Lineáris, Polinomikus, Gauss-i]
	\item \textbf{Generatív modellek} [Naive Bayes, Gauss-i keverékek]
	\item \textbf{Ajánló rendszerek} [Kollaboratív, metaadat-alapú]
	\item \textbf{Dimenziócsökkentés, Klaszterezés} [PCA, K-Means, Dbscan, Gauss-i]
	\item \textbf{Neurális hálózatok, Mélytanulás}
	\item \textbf{Megerősítéses tanulás}
\end{enumerate}
\end{frame}

\begin{frame}{Az órai környezet telepítése}
A csomagok és függőségek a \texttt{requirements.txt} fájlban vannak eltárolva. A csomagok telepítéséhez az Anaconda terminálban a következő parancsot kell lefuttatni:
\begin{block}{}
\texttt{pip install -r requirements.txt}
\end{block}
Ezzel az összes, a kurzus alatt vizsgált munkafüzetet futtatni lehet.
\end{frame}

\end{document}

























