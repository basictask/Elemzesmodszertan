% Settings for the default beamer theme
\documentclass[english, aspectratio=169]{beamer}
\usepackage[T1]{fontenc}
\usepackage[utf8]{inputenc}
\usepackage{tabularx}
\usepackage{babel}
\usepackage[ruled,vlined]{algorithm2e}
\SetAlgorithmName{Algoritmus}{algoritmus}{List of Algorithms}
\setcounter{secnumdepth}{3}
\setcounter{tocdepth}{3}

\makeatletter

\newcommand\makebeamertitle{\frame{\maketitle}}

% (ERT) argument for the TOC
\AtBeginDocument{%
  \let\origtableofcontents=\tableofcontents
  \def\tableofcontents{\@ifnextchar[{\origtableofcontents}{\gobbletableofcontents}}
  \def\gobbletableofcontents#1{\origtableofcontents}
}

% Theme settings
\usetheme{Frankfurt}
\usecolortheme{default}
\usefonttheme[onlymath]{serif}

% Template settings
\setbeamertemplate{navigation symbols}{}
\setbeamertemplate{blocks}[rounded][shadow=false]
\setbeamertemplate{title page}[default][colsep=-4bp, rounded=true, shadow=false]
\makeatother

% Define a custom darker red color
\definecolor{DarkerRed}{RGB}{139,0,0} % Adjust the RGB values as needed

% Use the newly defined color in Beamer theme elements
\setbeamercolor{structure}{fg=DarkerRed} % Changes basic structural elements to Darker Red
\setbeamercolor{title in head/foot}{bg=DarkerRed} % Changes the title in header/footer to Darker Red


\begin{document}

% Title page
\section{Bevezetés}
\title[]{Üzleti Elemzések Módszertana}
\subtitle{Tantárgyi útmutató}
\author[Kuknyó Dániel]{Kuknyó Dániel\\Budapesti Gazdasági Egyetem}
\date{2023/24\\2.félév}
\makebeamertitle

\begin{frame}{A jegy összetétele}
\begin{itemize}
	\item Összesen 100p gyűjthető
	\item Két elméleti teszt, egyenként 10-10p: az előadáson elhangzottakból
	\item 80p csoportos projekt feladat:
	\begin{itemize}
		\item Max. 3 fős csapatok
		\item Téma kiválasztása és kidolgozása
		\item A projekt összetétele:
		\begin{itemize}
			\item Jelenléti előadás
			\item Elemzési csővezeték (program)
			\item Dokumentáció
		\end{itemize}
	\end{itemize}
\end{itemize}
\end{frame}

\begin{frame}{Ponthatárok}
\begin{itemize}
	\item A teljes pontszám az elméleti tesztek és a beadandók összege
	\item Ponthatárok:\\
	$90 \leq x \leq 100 \rightarrow 5$\\
	$80 \leq x < 90 \rightarrow 4$\\
	$70 \leq x < 80 \rightarrow 3$\\
	$60 \leq x < 70 \rightarrow 2$\\
	$x < 60 \rightarrow 1$
	\item Ha a Hallgatónak nem sikerül elérnie a minimum követelményt, tehet vizsgát
\end{itemize}
\end{frame}

\begin{frame}{A téma kidolgozása}
\begin{itemize}
	\item A témáktól el lehet térni olyan algoritmusokra, modellfajtákra, amelyek az óra tárgyát nem képezték. Ha pl. valaki be szeretne mutatni egy transzfertanulási esettanulmányt, teljes értékű munkának számít! (Természetesen a módszertani témán belül maradva)
	\item A munka egyedisége és eredetisége elvárt és ellenőrzött.
	\item Konzultációs időre lehetősége van minden csapatnak, ahol feltehetik a kérdéseiket, tanácsot kérhetnek. Ez kb. 20-30 perc, Teams felületen.
	\item Ha nem teljesen világos a projekt terjedelme (mit, mennyit, milyen részletesen...) konzultáció keretein belül közösen segítünk meghatározni.
\end{itemize}
\end{frame}

\begin{frame}{A téma kidolgozása}
\begin{itemize}
	\item Terjedelem: min. 4-5 oldal képek nélkül Latex-ban vagy Lyx-ben megírva\par\smallskip
	\item Tartalma:
	\begin{itemize}
		\item Az elemzés célja, megválaszolandó kérdések
		\item Adatforrások bemutatása, tisztítási módszerek és ezek elméleti alapjai
		\item A felhasznált módszerek és ezek elméleti alapjai
		\item A hiperparaméter optimalizálás módszerei és eredményei
		\item A kutatási eredmények ismertetése, vizuális bemutatása
		\item Javaslatok és további fejlesztés lehetőségei, más rendszerekkel való kapcsolatok
	\end{itemize}\par\smallskip
	\item Csatolni kell [MintaPeter\_VincsEszter\_TesztElek.zip]:
	\begin{itemize}
		\item A felhasznált adatokat, és a linket ahonnan az adatok származnak [.csv, .xlsx]
		\item A programkódot, ami az elemzést megvalósítja [.py, .ipynb stb...]
		\item Az elemzéshez tartozó dokumentációt [.pdf]
		\item Az elemzésről szóló bemutatót [.pptx stb...]
	\end{itemize}
\end{itemize}
\end{frame}

\begin{frame}{Választható témák}
\begin{enumerate}
	\item Regresszió [Lineáris, Logisztikus] és Gradiens ereszkedés [Kötegelt, Sztochasztikus, Mini-kötegelt]
	\item Döntési fák, véletlen erdő [Bagging, Pasting, optimalizálás] Turbózás, együttes tanulás [Adaboost, GBM, XGBoost]
	\item Regularizált modellek [Lasso, Ridge, Elasztikus hálók]
	\item Tartó vektor gépek [Lineáris, Polinomikus, Gauss-i]
	\item Generatív modellek [Naive Bayes, Gauss-i keverékek]
	\item Ajánló rendszerek [Kollaboratív, metaadat-alapú]
	\item Dimenziócsökkentés és klaszterezés [PCA, K-Means, Dbscan, Gauss-i]
	\item Neurális hálózatok és mélytanulás
	\item Megerősített tanulás
\end{enumerate}
\end{frame}

\end{document}

























